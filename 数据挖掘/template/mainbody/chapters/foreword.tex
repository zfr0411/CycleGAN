\chapter{\heiti \label{ch1}绪论}
\section{\heiti 选题背景与意义}

Handling another style of scene with different styles 
is a time-consuming and laborious task and requires 
a lot of professional drawing skills. 
In order to obtain a high quality picture, 
the original author must carefully draw each line 
and paint each color area of the target scene. 
It appears that existing art editing software and 
algorithms with standard features do not produce 
satisfactory comic effects. Therefore, 
if professional technology can automatically convert 
one style of photo into another, 
it is very helpful for the artist: 
it can save them a lot of time and 
let them focus on More meaningful and creative work
The goal of picture migration is to use 
a paired picture training data set to learn 
the mapping between the input picture 
and the output picture, such as pix2pix\cite{pix2pix}, 
however, there are no paired training data sets for many items.
The innovation of CycleGAN\cite{Cyclegan} is that can migrate 
picture content from the source domain to the target domain 
without paired training data.

\begin{enumerate}[(1)] 
	\item When CycleGAN\cite{Cyclegan} is training, it only needs to 
	input the image of the source domain and the image of the target domain.
	This does not require the source domain to match the image content of the target domain.
	
	\item Based on cyclegan's CariGAN\cite{cGAN}, 
	you can automatically convert live-action photos 
	into formally exaggerated comics without paired images.
	
	\item With CycleGAN, not only convert between two types of images, 
	but also convert between two objects, 
	such as translating one person into another.
	
	
\end{enumerate}


\section{\heiti 国内外研究现状和相关工作}
随着国内外研究者们对图像迁移越来越关注,
出现了一系列的Cyclegan算法。
在这一小节,我们简短的介绍四条相关工作 
(Generative Adversarial Networks (GANs)、
Image-to-Image Translation、
Unpaired Image-to-Image Translation、
Cycle Consistency)
和其国内外研究现状。
\subsection{\heiti Generative Adversarial Networks (GANs)}
Generative Adversarial Networks (GANs)\cite{GAN,GAN1}
have achieved impressive results in image generation\cite{imagegen,imagegen1},
image editing \cite{imgaeedi}, 
and representation learning\cite{imagegen1,replearn}.
The key to the success of GAN is 
the concept of confrontational loss, 
which forces the generated image to be 
indistinguishable from the real image in principle.

\subsection{\heiti Image-to-Image Translation}
The idea of image to image conversion 
is to use a non-parametric texture model\cite{texmodel}
 on a single input-output training image pair. 
 The latest method uses the data set of 
 input and output checks to learn the 
 parameter conversion function through CNN\cite{CNN}. 
 Our approach is based on the "pix2pix"\cite{pix2pix} 
 framework of Isola et al\cite{Isola}.
 It uses conditional generation of the network 
 to learn the mapping from input to output images.


\subsection{\heiti Unpaired Image-to-Image Translation}
The similarity function between the predefined inputs 
and outputs that does not depend on any particular task, 
nor does it assume that we assume that the inputs 
and outputs must be in the same low-dimensional 
embedded space. This makes our approach a 
universal solution for many visual and graphical tasks.

\subsection{\heiti Cycle Consistency}
In target tracking, language translation, 
3D shape registration and so on. 
There are also applications in CNN. 
A similar loss was introduced to push G and F 
to agree with each other.

\section{\heiti 国内外研究现状}
\begin{enumerate}[(1)] 
	\item In 2018, researchers from Tsinghua University,
	 City University of Hong Kong and Microsoft recently 
	 proposed CariGAN\cite{cGAN}, which can automatically 
	 convert live-action photos into formally 
	 exaggerated cartoons without paired images.
	
	\item 
	
	
\end{enumerate}



% \newpage
